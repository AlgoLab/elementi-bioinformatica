\usepackage[italian]{babel}
\usepackage[utf8]{inputenc}
\usepackage{pgf}
\usepackage{verbatim}

% \usepackage{euler}
\usepackage[T1]{fontenc}

\immediate\write18{sh ./vc}
\input{vc}

\author{Gianluca Della Vedova}
\title{Elementi di Bioinformatica}
\institute{Univ. Milano--Bicocca\\
  \texttt{http://gianluca.dellavedova.org}}
\date{\today, {\tiny revisione \VCRevision}}
\pgfdeclareimage[height=1cm]{university-logo}{logounimib}
\logo{\pgfuseimage{university-logo}}



% If you wish to uncover everything in a step-wise fashion, uncomment
% the following command:
% \beamerdefaultoverlayspecification{<+->}


\begin{document}

\begin{frame}
  \titlepage
\end{frame}


\begin{frame}\frametitle{Gianluca Della Vedova}
  \begin{itemize}
  \item
    Elementi di Bioinformatica
  \item
    Ufficio U14-2041
  \item
    \textsf{\small http://gianluca.dellavedova.org}
  \item
      \textsf{\small
      http://algolab.eu/didattica/elementi-di-bioinformatica/}
  \item
    \textsf{\small gianluca.dellavedova@unimib.it}
  \end{itemize}
\end{frame}

 \begin{frame}
   \frametitle{Consensus Clustering}
   \begin{block}{Problem}
     \alert{Input}: set $\Pi$ of partitions of $U$\\
     \alert{Output}: a partition $P$ of $U$\\
     \alert{Goal}: $P$ the best possible representative of $\Pi$
   \end{block}

   \begin{block}{Objective Function}
     \begin{itemize}
     \item
       Symmetric difference between two partitions =
       Pairs of elements clustered differently
     \item
       $d(P,\Pi)=\sum_{\pi\in \Pi}d(\pi,P)$
     \end{itemize}
   \end{block}
 \end{frame}



\begin{frame}\frametitle{Finestre}
  \begin{itemize}
  \item
    Editor (F5)
  \item
    Log (F6)
  \item
    Output (F7)
  \item
    Icona Esegui (F8)
  \end{itemize}
\end{frame}


%\begin{frame}[fragile]\frametitle{Data Step: da dati grezzi 2}
    %  \VerbatimInput{code/p3.sas}
    % Dati separati da spazi
%\end{frame}

\begin{frame}[fragile]\frametitle{Formati Alfanumerici}
  \verb!: $20.!

  Rappresenta il numero massimo di caratteri di un campo (in questo caso
  20). Il dato termina con il primo spazio.

  \verb!$20.!

  Esattamente 20 caratteri.
\end{frame}

\begin{frame}[fragile]\frametitle{Formati Numerici}
  \verb!10.3!

  Numero totale di caratteri/cifre

  .

  Numero di cifre decimali
\end{frame}


\begin{frame}[containsverbatim]\frametitle{Licenza d'uso}
  \small

  Quest'opera {\`e} soggetta alla licenza Creative Commons: Attribuzione-Condividi
  allo stesso modo 3.0.

  \verb+http://creativecommons.org/licenses/by-sa/3.0/+

  Sei libero di riprodurre, distribuire, comunicare al pubblico, esporre in
  pubblico, rappresentare, eseguire, recitare e modificare quest'opera
  alle seguenti condizioni:
  \begin{itemize}
  \item
    Attribuzione — Devi attribuire la paternit{\`a} dell'opera nei modi indicati
    dall'autore o da chi ti ha dato l'opera in licenza e in modo tale da non
    suggerire che essi avallino te o il modo in cui tu usi l'opera.
  \item
    Condividi allo stesso modo — Se alteri o trasformi quest'opera, o se la usi
    per crearne un'altra, puoi distribuire l'opera risultante solo con una licenza
    identica o equivalente a  questa.
  \end{itemize}
  \vspace*{1cm}
\end{frame}

\end{document}




%%% Local Variables:
%%% mode: latex
%%% TeX-PDF-mode: t
%%% TeX-master: "lucidi_bioinformatica_video"
%%% buffer-file-coding-system: utf-8
%%% End:
